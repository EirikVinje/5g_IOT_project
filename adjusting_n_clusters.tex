% This file was created with tikzplotly version 0.1.7.
\pgfplotstableread{data0 radiusx3d5 radiusx3d6 radiusx3d7 radiusx3d8 radiusx3d9 radiusx3d10
1 0.07544642857142857 0.10446428571428572 0.14419642857142856 0.19285714285714287 0.23973214285714287 0.29017857142857145
2 0.14821428571428572 0.21651785714285715 0.28125 0.36428571428571427 0.45848214285714284 0.5459821428571429
3 0.2200892857142857 0.32142857142857145 0.42142857142857143 0.5544642857142857 0.6816964285714285 0.778125
4 0.2736607142857143 0.40669642857142857 0.5522321428571428 0.6924107142857143 0.8191964285714286 0.8995535714285714
5 0.35446428571428573 0.5227678571428571 0.7008928571428571 0.828125 0.9102678571428572 0.9669642857142857
6 0.4263392857142857 0.6044642857142857 0.7575892857142857 0.8754464285714286 0.9508928571428571 0.9861607142857143
7 0.496875 0.6732142857142858 0.8075892857142857 0.903125 0.965625 0.9955357142857143
8 0.5790178571428571 0.79375 0.9316964285714285 0.9870535714285714 0.9991071428571429 1.0
9 0.6299107142857143 0.8334821428571428 0.9549107142857143 0.9941964285714285 1.0 1.0
10 0.7178571428571429 0.8991071428571429 0.9790178571428572 0.9977678571428571 1.0 1.0
}\dataZ

\begin{tikzpicture}


\begin{axis}[
title=Adjusting n\_clusters for different number of radiuses,
xlabel=n\_clusters,
ylabel=Coverage ratio
]
\addplot+ [mark=none] table[y=radiusx3d5] {\dataZ};
\addlegendentry{radius=5}
\addplot+ [mark=none] table[y=radiusx3d6] {\dataZ};
\addlegendentry{radius=6}
\addplot+ [mark=none] table[y=radiusx3d7] {\dataZ};
\addlegendentry{radius=7}
\addplot+ [mark=none] table[y=radiusx3d8] {\dataZ};
\addlegendentry{radius=8}
\addplot+ [mark=none] table[y=radiusx3d9] {\dataZ};
\addlegendentry{radius=9}
\addplot+ [mark=none] table[y=radiusx3d10] {\dataZ};
\addlegendentry{radius=10}
\end{axis}
\end{tikzpicture}
